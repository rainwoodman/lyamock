\documentclass{paper}
\usepackage{units}
\usepackage{amsmath}
\begin{document}
\newcommand{\diff}{\,\mathrm{d}}
\section{Cosmology}
\subsection{Hubble constant}
\[
    H(a) = H_0 a^{-1.5} \left[\Omega_M + 
           (1 - \Omega_M - \Omega_L) a +
       L a^3 \right] ^ {0.5}  .
\] 
The dimensionless Hubble constant
\[
    E(a) = \frac{H(a)}{H_0} .
\]

\subsection{Comoving distance}
\[
    D_c = D_H \int \frac{1}{E(z)} \diff z 
        = \int \frac{1}{a E(a)} \diff \log a,
\]
where Hubble distance $D_H = \frac{c}{H_0}$.
\subsection{Growth Factor}
\[
    \Delta^+(a) = E(a) 
\int_{-\infty}^{\log a} \left[a E(a)\right]^{-3} a \diff \log a .
\]
Note that this does not contain the 2.5 factor people
usually use. The factor will cancel out when we calculate
the growth factor relative to today,
$\Delta^+_0(a) = \Delta^+(a) / \Delta^+(1.0)$.
\subsection{Displacement to velocity}
This is used to convert displacement to velocity. 
\[
    F_\Omega(a) = \left[\frac{\Omega_M}{a^3 E^2(a)}\right]^{0.6} .
\]
Avarind suggests that the magic 0.6 is related to $\sigma_8
\Omega_M ^ 0.6$ that shows up on WMAP7 website.
\[
    \mathbf{v} = a H(a) F_\Omega(a) \mathbf{\Psi} .
\]
Note that in GADGET the velocity saved in snapshot file is 
$\sqrt{a}\mathbf{v}$.

\section{Log Normal}
See Bi \& Davidsen, 1997. To avoid negative density.
\[
    \delta_b (x) = \exp\left[
        \delta_0 (\mathbf{x}) 
      - \frac{\left<\delta^2_0\right>}{2}
    \right],
\]
where $\delta_0$ is the baryon gaussian derived from 
 matter gaussian,
\[
    \delta_0(\mathbf{k}) = \frac{\delta_\mathrm{DM}
    (\mathbf{k})}{1 + x^2_b k^2}.
\]
The comoving scale, $x_b$ is $\sim \unit[0.1]{h^{-1}Mpc}$.
For $k \ll \frac{2 \pi}{x_b} \approx \unit[10]{h
    Mpc^{-1}}$, we have
\[
    \delta_0(\mathbf{k}) = \delta_\mathrm{DM}(\mathbf{k}) .
\]
The displacement field shall be obtained from the gaussian
DM field, assuming gas tracks DM.
\[
    \nabla \cdot \mathbf{\Psi}(\mathbf{x}) =
    -\delta_\mathrm{DM}(\mathbf{x}) .
\]
As the field is already available in k-space, just need to
follow the usual approach,
\[
    \mathbf{\Psi}(\mathbf{k}) = 
    -\frac{\imath \mathbf{k}}{k^2} \delta_\mathrm{DM}(\mathbf{k}) .
\]
\section{Mean transimission flux}
Faucher-Giguere is the person behind this. A fit from Rupert
says
\[
        \log_{10} \tau_\mathrm{eff} = 
          -2.910 + 4.164 \log_{10} (1 + z),
\]
where $\tau_\mathrm{eff} = - \log \left<F\right> $.
\section{FPGA}
see arxiv 9709303 Weinberg et al 1997 and also  LeGoff et
al. 2011 (DOI: 10.1051/0004-6361/201117736).

The optical depth $\tau$ is approximately,
\[  \tau(a) = 
    A_0(a)
    \underbrace{
        \exp \left(\beta \delta_b \right)
    }_{A_1(a)}
    \underbrace{
        \left(1 + 
        \frac{\diff ( \hat{r} \cdot \mathbf{v})}
        {H(a) \diff r } \right)^{-1}
    }_{A_2(a)}
    \underbrace{
        \left(\frac{\Gamma}{\unit[10^{-12}][s^{-1}]}
        \right)^{-1}
    }_{A_3(a)}
    ,
\] where $A_1$ accounts for density fluctuation, 
         $A_2$ for redshift distortion, 
         $A_3$ for photo-ionization, 
   and we have grouped all other factors in Weinberg paper to $A_0(a)$, 
   which will be decided by fitting the mean flux or 1D
   powerspectrum to observation.

   $A_1$ appears different from Weinberg paper. However, the
   expansion gives
   \[ 
     \exp \left( \beta \delta_b \right)
     = (1 + \delta + \dots)^\beta
     \approx \left(\frac{\rho_b}{\bar{\rho_b}}\right)^\beta
   \]


\end{document}
