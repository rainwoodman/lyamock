\documentclass{paper}
\usepackage{units}
\usepackage{amsmath}
\usepackage{graphicx}

\begin{document}
\newcommand{\diff}{\,\mathrm{d}}
\newcommand{\mpch}{\,\unit{h^{-1}Mpc}}
\newcommand{\kpch}{\,\unit{h^{-1}Kpc}}
\title{A method to generate Quasar Correlated Lyman-$\alpha$
    forest mocks}
\author{Yu Feng}
\maketitle

\section{Summary}
    The quasars in the mocks used in current Lyman-$\alpha$ forest analysis 
    do not correlat with the density field. 

    This article describes a method to produce mocks with correlated quasar and forest.

    For a given cosmology ($\Omega$s), a survey
    profile (quasar number density per redshift and sky
    mask), the method produces a catelogue of quasars (redshift, RA
    and DEC), and the optical depth $\tau$ or transmission fraction $F$ of 
    forest pixels on uniform log-$\lambda$ pixels, binned into SDSS
    pixels. Redshift distortion is included. 
    
    The method assumes \textbf{linear theory} in the underlining
    matter and velocity field. \textbf{Lognormal transformation} (peak-background
    split) is used to convert the matter field to local quasar
    number density. \textbf{Fluctuating Gunn-Peterson
    approximation} in addition to lognormal transformation is used to
    convert the linear matter field to optical depth that
    matches the mean and intrinsic variance of the
    transmission fraction.
    
    The density field from which quasars are sampled is 
    smoothed at $R_Q = 10\mpch$.
    Forest field is smoothed at $R_\mathrm{LN} = 250\kpch$. The correlation
    of forest is cut at $5\mpch$. The velocity field used in
    redshift distortion of both (quasar and forest) is
    cut at $5\mpch$.

    The Gaussian modes in $k$ space for the quasar and forest
    density field are identical $k$-space Gaussian random field
    up to the required scale.  This guareentee the cross
    correlation between quasars and the forest.

    The code for generating mocks is in good shape; and its behavior
    fairly well understood.
    Figure \ref{fig:handwave} shows the correlation function measured
    from 50 mocks on a DR10 foot-print survey.

    We have been trying to verify the mocks with a RSD
    measurement with the covariance matrix. 
    \begin{figure}
        \includegraphics[width=\columnwidth]{../mocks-10mpcqso/handwaving}
        \caption{Correlation functions on 50 Mocks with a
        fake cosmology. Shaded area covers the scale
        smoothing contaminates the linear theory correlation.  }
        \label{fig:handwave}
    \end{figure}

\subsection{Notations}
\begin{itemize}
    \item $P(k, z)$
        linear theory matter power spectrum at $z$; input;
    \item $D(z)$
        growth factor relative to $z=0$; input;

    \item $x_0$ positions of the observer, conveniently
        sitting in the center of the box; input;

    \item $y_i$ positions of the $i$-th quasar; derived;

    \item $\delta(x; R)$ 
        Gaussian random field, smoothed to scale $R$;
        smoothing is done in Fourier space with a kernel 
        $(1 + (kR)^2)^{-1}$; derived;
    \item $\delta|_{k_1}^{k_2}(x)$ 
        Long range Gaussian random field; with power
        spectrum cut between $k_1$ and $k_2$; derived;

    \item $\eta_\mathrm{LN}(x; z)$ the lognormal
        transformation of the over-density field; derived;

    \item $b_Q(z)$ quasar bias; input;

    \item $A(z)$ $B(z)$ parameters in FGPA for 
        the Lyman$\alpha$ forest; $A$ corresponds to the
        column density of HI clouds; $B$ corresponds to the 
        equation of state of IGM; derived;

    \item $\left<n_Q(z)\right>$ quasar number density as
        function of redshift, obtained from survey / QLF;
        input;
    \item $\left<F(z)\right>$ the model of mean transmission
        fraction; input;
    \item $\left<\sigma_F(z)\right>$ the model of variance of
        transmission fraction; input;

\end{itemize}

\section{Split Power Spectrum}
    Over-density field $\delta(r)$ is the Fourier transform of
    the power spectrum $P(k)$. For example, if $P(k)$ is cut to 
    long range and short range at $k_0$, 

    \begin{eqnarray*}
        P_l(k; z) = \Theta(k - k_0)P(k; z) \\
        P_s(k; z) = (1 - \Theta(k - k_0)) P(k; z);
    \end{eqnarray*}
    then the over-density field is also cut into two pieces
    \[
        \delta(x) = \delta|_{0}^{k_0}(x) +
        \delta|_{k_0}^{\infty}(x).
    \]
    Notice that in practice, the Fourier transform is discrete.
    Some form of interpolation over the large scale over-density field
    shall be used; this interpolation shall be corrected in the small scale
    field power spectrum which is not currently done.

\section{Multiple realization at different scale}
    We will sample the density field at different
    resolution (with different size of discrete Fourier transforms).    
    One issue is to maintain the consistency between different
    realizations. 

    The issue can be solved in the $K$-space.  We populate the K-space
    Gaussian field with random numbers sampled
    from identically initialized pseudo-random-number generators, in
    the order from small $K$ [large scale] to large $K$ [small scale].
    The algorithm is similar to that used in \small{Gadget}
    / \small{N-GenIC}.

\section{Point-wise objects: Quasars}
    Quasar are point-wise objects. The problem we are
    solving is to sample biased point-like objects that
    from continuous density field. In practice these
    point-wise objects can be anything that have a bias, eg, 
    DLA or Lyman-limit systems.

    Design goals:
    \begin{itemize}
        \item the algorithm shall be local; this eliminates the 
              communication between sub-boxes;
        \item the resolution of density field shall be coarse; 
              because eventually every sample point in the density
              field will be scanned;
        \item at linear scale, the algorithm shall agree
              with linear theory, yielding the correct bias,
              etc.
    \end{itemize}

    Inputs:
    \begin{enumerate}
        \item 
            We generate a realization of the density field 
        with $R_Q = 10\,\mpch$ smoothing $\delta(x; R_Q)$ on
        a $10\,\mpch$ mesh. 
        \item $\left<n_Q(x, z)\right>$, 
            the mean number density of quasars; this density is
            corrected for the sky coverage.
    \end{enumerate}

    We require the expected number density of quasars in a given $R_Q$
    cell, at redshift $z$ to be
    \[
        E[n_Q(x; z)] = \left<n_Q(x, z)\right> 
        \exp \left\{b(z) \left( 
        \delta(x; R_Q) - \frac{1}{2}\left<\delta(x;
        R_Q)^2\right>
    \right)\right\}.
    \]

    The transformation is
    motivated by peak-background split; however notice that
    the peak-background split condition is not, strictly
    speaking, satisfied in this case.

    The parameter $b(z)$ controls the bias. Our numerical
    experiment shows that the bias of quasars generated with
    the lognormal method roughly follows $b(z)$, up to
    $b(z)=4$, when $R_Q$ is around $10\,\mpch$. For a crude
    approximation, we set $b(z) = b_Q(z)$.

    The expected number of quasars in a cell is given by 
    \[
        E[N_Q(x; z)] = R_Q^3 n_Q(x; z).
    \]

    The actual number of quasars in a cell is a random
    variable. We assume a Poisson distribution with mean at
    $E[N_Q(x; z)]$. Simple math shows that the distribution 
    does not contaminate the correlation function, as long
    as the numbers in different cells are IIDs.

    The position of a quasar with in a cell ($r_{ab}$) is then draw from a
    uniform distribution. Simple math shows that the
    positioning within a cell has no effect on the
    correlation function as long as the positions are IIDs.

    \[
        y_{a, b} = x_a + r_{ab}.
    \]

    As a final step, the Quasars that are not in the sky coverage are
    removed.

\subsection{Generating $\delta(x; R_\mathrm{Q})$}
    We use two pieces of power spectrum; 
    split at $k = 2\pi / \left\{600\, \mpch \right\}$.

\section{Continuous: Lyman $\alpha$ Forest}
    After the position of quasars is determined, we can draw
    sightlines from the position of quasars $y_i$ 
    to the observer $x_0$. 
    
    Unlike the discrete point-wise quasars, the
    ``over'' transmission fraction of forest ($\delta_F$) is
    a continuous field; the pixels on the sightlines simply
    sample this continuous at some given scale.

    We apply a lognormal transformation, then the Fluctuating Gunn Peterson
    Approximation (FGPA) to convert the correlated Gaussian density
    field $\delta(x)$ to a optical depth field.

    The lognormal transformation is 
    \[
        \eta_\mathrm{LN}(x, z) = \exp\left\{
        D(z) \delta(x; R_\mathrm{LN}) - \frac{1}{2} D(z)^2
        \left<\delta(x; R_\mathrm{LN})^2\right> 
        \right\},
    \]

    The lognormal transformation is applicable on a comoving
    scale $R_\mathrm{LN}$ of around $200\,\kpch$. In
    Bi \& Davidson, they choose a scale where the variance of the smoothed density field 
    is around 1.0. We want to follow the
    same variance. 
    
    In contrast to Bi \& Davidson, our density field has redshift 
    evolution: $\delta(x; z) = D(z) \delta(x; z=0)$. We choose a 
    smoothing scale where the variance between redshift $z=2.0$ 
    and $z=3.0$ is close to 1.0; the scale is $R_\mathrm{LN} =
    250\,\kpch$.

    The formula for FGPA is
    \[
        T(x, z; R_\mathrm{LN}) = 
          R_\mathrm{LN} A(z) 
            \left(\eta_\mathrm{LN}(x,
          z)\right) ^ {B(z)}.
    \]

    The two redshift dependant free parameters are $A(z)$
    and $B(z)$. $A(z)$ controls the overall transmission
    fraction at a given redshift, while $B(z)$ controls
    the variance of the transmission fraction. These
    parameters can be solved by requiring the mocks to match
    up with the models: (fitting formula listed in Lee et al 2013)

    \begin{enumerate}
        \item The mean transmission fraction $\left<F\right>(z)$;
        \item The intrinsic variance $\sigma_F(z)$.
    \end{enumerate}

    We cannot perform the fitting yet, because the 
    optical depth field along line of sight $T(x, z; R_\mathrm{LN})$ 
    is smoothed to $R_\mathrm{LN} = 250 \,\kpch$, a scale
    very different from the length scale of a pixel in the survey.

    To match the length scale, we resample $T(x, z; R_\mathrm{LN})$ 
    onto the uniform log-$\lambda$ grid of the SDSS pixels. 
    The integral for the $j$-th pixels on the $i$-th
    quasar is
    \[
        \tau_i(j) = \int_{r_{j1}}^{r_{j2}} 
        R_\mathrm{LN}^{-1} T(x_i(r), z_i(r); R_\mathrm{LN}) d r, 
    \]
    where $r_{j1}$ is the comoving distance at the beginning of 
    the $j$-th pixel, $r_{j2}$ being the end.
    $x_i(r)$ is the position of along the line of sight of
    $i$-th quasar, relative to the observer at $x_0$. 

    After resampling, we solve for $A(z)$ and $B(z)$ such
    that the transmission fraction 
    of $F = \exp -\tau_i(j)$  matches the constraints on
    $\left<F\right>(z)$ and $\sigma_F(z)$. 

    $B(z)$ corresponds to the FGPA parameter 
    $b = 2-0.7(\gamma-1)$, where $\gamma$ describes the equation of 
    state of IGM.  Allowing a running $B(z)$ 
    is equivalent of allowing a running equation of state of IGM.
    $B(z)$ varies slowly with redshift from 1.2 to 2.0 from
    $z=2.0$ to $z=4.0$. A canonical value used in prevoius
    literature is $\gamma = 1.6$, $b=1.58$.

    Interestingly, $A(z)$ can be fit by an exponential function of $1/(1+z)$
    $B(z)$ can be fit by a second order polynomial of
    $1/(1+z)$. The origin of these fits is not understood.
    See Figure \ref{fig:FGPA-AB}. The fitted mean
    and intrinsic variance of forest pixels are shown in
    Figure \ref{fig:FGPA-MV}.

    \begin{figure}
      \includegraphics{../mocks-10mpcqso/AB}
      \caption{Parameters in FGPA. $A(z)$ and $B(z)$. See text.}
      \label{fig:FGPA-AB}
    \end{figure}

    \begin{figure}
      \includegraphics{../mocks-10mpcqso/MV}
      \caption{Fitting to the transmission model. $\left<F\right>$ and
$\sigma_F(z)$.}
      \label{fig:FGPA-MV}
    \end{figure}

\subsection{Generating $\delta(x; R_\mathrm{LN})$}
    In practice, the over-density field at lognormal scale is
    generated with the power spectrum split into 3 pieces.
    The first split is the same as the one in $\delta(x, R_Q)$. 
    The second split is at around $R_S = 2\pi / k_2 = 5
    \unit{Mpc/h}$.

    The total over-density field contains three pieces,
    \[
        \delta(x; R_\mathrm{LN}) =
        \delta|_{k_1}^{k_\mathrm{max}}(x) +
        \delta|_{k_2}^{k_1}(x) +
        \delta|_{k_\mathrm{min}}^{k_2}(x, R_\mathrm{LN}).
    \]

    We take an approximation on the third piece. We do not
    generate(spatially) correlated Gaussian field on the
    very small scale; rather, we replace the their piece
    with a uncorrelated Gaussian with the same variance,
    \[
        \delta|_{k_\mathrm{min}}^{k_2}(x, R_\mathrm{LN})
        \approx g \sim N(0, \left<\delta|_{k_\mathrm{min}}^{k_2}(x,
        R_\mathrm{LN})^2\right>).
    \]
    
    The approximation is different from LeGoff et al who
    used templates with correlated small scale pixels.
    We feel that since there is no reason to believe our
    approach (linear theory + lognormal + FGPA) correctly capture the
    short range correlation of forest pixels, there is no need to
    include the small scale correlation ($r < R_S$), which adds on to the
    computational complexity.

\section{Redshift Distortion}
Along with the initial Gaussian field at $R_S$, we also generate the
linear theory displacement field $\Phi(x, z)$. Three components $x$,
$y$, $z$ are generated with a simple differential kernel in $K$ space.
The redshift evolution is linear

\[
    \Phi(x, z) = D(z) \Phi(x, z=0).
\]

The linear theory displacement $\Phi(x, z)$ is converted to RSD
displacement via the Kaiser formula,
\[
    \mathbf{\Psi}(x, z) = F_\Omega(a) \mathbf{\Phi}(x, z),
\]
where
\[
    F_\Omega(a) = \left[\frac{\Omega_M}{a^3 E^2(a)}\right]^{0.6} .
\]

The redshift distortion displacement $\mathbf{\Psi}(x, z)$ is projected along the
sight line. The position of the sample points on the $R_\mathrm{LN}$
scale is shifted by $\Psi$ before the resampling to spectra pixels.

The location of quasars is also shifted by the same RSD displacement
field.

\appendix
\section{Fitting}
\begin{figure}
    \includegraphics[width=\columnwidth]{../mocks-10mpcqso/000/bestfit}
    \caption{Bestfit of 000}
\end{figure}
\begin{figure}
    \includegraphics[width=\columnwidth]{../mocks-10mpcqso/001/bestfit}
    \caption{Bestfit of 001}
\end{figure}
\begin{figure}
    \includegraphics[width=\columnwidth]{../mocks-10mpcqso/002/bestfit}
    \caption{Bestfit of 002}
\end{figure}
\begin{figure}
    \includegraphics[width=\columnwidth]{../mocks-10mpcqso/bestfit-000}
    \caption{Collage of fits 000-009}
\end{figure}
\begin{figure}
    \includegraphics[width=\columnwidth]{../mocks-10mpcqso/bestfit-010}
    \caption{Collage of fits 010-019}
\end{figure}
\begin{figure}
    \includegraphics[width=\columnwidth]{../mocks-10mpcqso/bestfit-020}
    \caption{Collage of fits 020-029}
\end{figure}

\section{Useful Formula}
\subsection{Hubble constant}
\[
    H(a) = H_0 a^{-1.5} \left[\Omega_M + 
           (1 - \Omega_M - \Omega_L) a +
       L a^3 \right] ^ {0.5}  .
\] 
The dimensionless Hubble constant
\[
    E(a) = \frac{H(a)}{H_0} .
\]

\subsection{Comoving distance}
\[
    D_c = D_H \int \frac{1}{E(z)} \diff z 
        = \int \frac{1}{a E(a)} \diff \log a,
\]
where Hubble distance $D_H = \frac{c}{H_0}$.
\subsection{Growth Factor}
\[
    \Delta^+(a) = E(a) 
\int_{-\infty}^{\log a} \left[a E(a)\right]^{-3} a \diff \log a .
\]
Note that this does not contain the 2.5 factor people
usually use. The factor will cancel out when we calculate
the growth factor relative to today,
$\Delta^+_0(a) = \Delta^+(a) / \Delta^+(1.0)$.
\subsection{Displacement to velocity}
This is used to convert displacement to velocity. 
\[
    F_\Omega(a) = \left[\frac{\Omega_M}{a^3 E^2(a)}\right]^{0.6} .
\]

\[
    \mathbf{v} = a H(a) F_\Omega(a) \mathbf{\Psi} .
\]

The apparent displacement due to redshift distortion in
comoving units is therefore,
\[
    \Delta r_\mathrm{RSD} = F_\Omega(a) \mathbf{\Psi} .
\]

A side note: in GADGET the velocity saved in snapshot file is 
$\sqrt{a}\mathbf{v}$.

\subsection{FPGA}
see arxiv 9709303 Weinberg et al 1997 and also  LeGoff et
al. 2011 (DOI: 10.1051/0004-6361/201117736).

The optical depth $\tau$ is approximately,
\[  \tau(a) = 
    A_0(a)
    \underbrace{
        \exp \left(\beta \delta_b \right)
    }_{A_1(a)}
    \underbrace{
        \left(1 + 
        \frac{\diff ( \hat{r} \cdot \mathbf{v})}
        {H(a) \diff r } \right)^{-1}
    }_{A_2(a)}
    \underbrace{
        \left(\frac{\Gamma}{\unit[10^{-12}][s^{-1}]}
        \right)^{-1}
    }_{A_3(a)}
    ,
\] where $A_1$ accounts for density fluctuation, 
         $A_2$ for redshift distortion, 
         $A_3$ for photo-ionization, 
   and we have grouped all other factors in Weinberg paper to $A_0(a)$, 
   which will be decided by fitting the mean flux or 1D
   powerspectrum to observation.

   $A_1$ appears different from Weinberg paper. However, the
   expansion gives
   \[ 
     \exp \left( \beta \delta_b \right)
     = (1 + \delta + \dots)^\beta
     \approx \left(\frac{\rho_b}{\bar{\rho_b}}\right)^\beta
   \]


\end{document}
